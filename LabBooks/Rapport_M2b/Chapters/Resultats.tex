%!TEX root = main.tex
%!TeX TS-program = pdflatex
%!TeX encoding = UTF-8 Unicode
%!TeX spellcheck = fr
%!BIB TS-program = biber
% -*- coding: UTF-8; -*-
% vim: set fenc=utf-8
% Chapter Template

\chapter{Résultats} % Main chapter title
 
\label{Résultats} % Change X to a consecutive number; for referencing this chapter elsewhere, use \ref{ChapterX}

%----------------------------------------------------------------------------------------

\section{Résultats escomptés}
\label{result_escompt}

Au terme du stade de développement actuel notre modèle devrait être capable, lorsque nous lui fournissons une image bruitée de $128\times 128$ pixels contenant un chiffe MNIST placé au hasard, de produire une carte de probabilités représentant une prédiction précise de la position du stimulus dans l'espace.
A chaque point de l'espace devrait donc correspondre une valeur comprise dans l'intervalle $[0,1]$, pouvant être traduite par la certitude du modèle de la présence d'un stimulus en ce point.
La prédiction du modèle devrait ainsi comprendre au moins une zone chaude (une seule lors d'une prédiction optimale), c'est à dire une région circulaire (dépendant de la forme des champs récepteurs) où sont rassemblées les valeurs de probabilités les plus élevées, à l'image de ce que l'on peut observer dans les labels (figure~\ref{fig:accuracy_128_LP}).
Selon les caractéristiques de notre filtre rétinien, l'acuité du modèle et donc la certitude de ses prédictions devraient diminuer avec l'excentricité par rapport au centre de la fovéa.
Ainsi lorsqu'on éloigne le stimulus de cette dernière, la zone chaude devrait s'étendre (révélant une prédiction moins précise) pendant que les probabilités qu'elle contient diminuent pour se rapprocher du seuil définissant le hasard (révélant une certitude plus faible). \autocite{Freeman2011, Werner2014} \\
Une fois cette prédiction de la position réalisée, le modèle devrait pouvoir l'utiliser pour réaliser une saccade rapprochant la position prédite de la cible du centre de la fovéa.
A noter qu'en l'absence d'organe perceptif physique, les saccades correspondent à un décalage spatial de l'image utilisée en entrée.
Enfin, après avoir supposément placé la cible sur sa fovéa ou tout du moins proche de celle-ci, le modèle pourra réaliser une classification dans la partie centrale de son champs visuel.
Si la prédiction de la position de la cible a été correcte et donc que la bonne partie de l'environnement visuel a été placé sur la fovéa, alors le modèle devrait être capable de classifier correctement le chiffre qu'on lui présente. \autocite{Werner2014}

\section{Résultats préliminaires}

Il est possible d'observer sur la figure~\ref{fig:prediction} une diminution de la valeur du coût d'environ 50\% lors des itérations de l'apprentissage automatisé. 
Une diminution de cette valeur est le signe que l'apprentissage a bien lieu et plus elle est importante, plus l'apprentissage devrait être efficace.
Notons qu'après une décroissance rapide (itérations 0-200) de cette valeur, nous pouvons rapidement observer (itérations 200-500) un ralentissement de celle-ci, phénomène couramment observé dans les modèles d'apprentissage automatique. \\
Lorsque, durant la phase d'évaluation, nous fournissons à notre modèle une image bruitée et observée au travers d'un filtre rétinien, celui-ci est capable de l'intégrer pour produire un vecteur contenant un ensemble de valeurs comprises dans l'intervalle $[0,1]$.
Ce vecteur, à partir duquel il est possible de réaliser par reconstruction un graphique (figure~\ref{fig:prediction}), présente une ou plusieurs zones chaudes où sont rassemblées les valeurs les plus importantes. 
Notons que la reconstruction graphique que nous utilisons modifie les valeurs des données, entraînant la perte de l'intervalité $[0,1]$ de celles-ci.
Bien que lors de l'écriture de ce rapport, aucune analyse quantitative n'ait été réalisée, la position de ces zones chaudes semble concorder avec la position réelle de la cible dans l'image fournie en entrée.
Il a toutefois été observé des erreurs importantes de prédictions lors de certains essais. \\
De même, bien que la réalisation d'une saccade pour se rapprocher de la position détectée de la cible ainsi que la classification de la région nouvellement observée soient exposées dans la partie~\ref{result_escompt}, ces fonctionnalités ne sont pas implantées lors de l'écriture de ce rapport.