% Chapter 1

\chapter{Introduction} % Main chapter title
%\addchaptertocentry{Introduction} 
\label{Introduction} % For referencing the chapter elsewhere, use \ref{Chapter1} 

%----------------------------------------------------------------------------------------

% Define some commands to keep the formatting separated from the content 
\newcommand{\keyword}[1]{\textbf{#1}}
\newcommand{\tabhead}[1]{\textbf{#1}}
\newcommand{\code}[1]{\texttt{#1}}
\newcommand{\file}[1]{\texttt{\bfseries#1}}
\newcommand{\option}[1]{\texttt{\itshape#1}}

%----------------------------------------------------------------------------------------

Au cours de l'histoire évolutive et sous la pression de la sélection naturelle, tous nos systèmes perceptifs ont tendu (et tendent encore) vers une optimisation de leurs performances, en fonction de nos besoins et de nos ressources.\\
Chez de nombreuses espèces dont la notre, la modalité perceptive principale est la vision.
L'ensemble de notre système visuel, de la rétine jusqu'aux aires cérébrales les plus associatives, a ainsi évolué pour obtenir le fonctionnement à la fois rapide et efficace qu'on lui connait aujourd'hui.
Pour cela, notre système visuel associe deux de ses caractéristiques: une acuité variable et des saccades oculaires. \autocite{Werner2014}\\

Le champs visuel peut être décrit en deux parties: la vision centrale et la vision périphérique. \autocite{Werner2014}\\
La vision centrale est soutenue anatomiquement par la fovéa, une région rétinienne comprenant exclusivement des cônes. Cette composition, couplée à une forte densité de photorécepteurs permet à cette région de présenter l'acuité visuelle (c'est à dire la précision avec laquelle les stimuli visuels pourront être analysés) la plus importante, ainsi qu'une bonne perception des couleurs. \autocite{Werner2014}\\
La composition et la densité en photorécépteurs de la rétine soutenant la vision périphérique change avec son excentricité par rapport à la fovéa, mais elle comprends majoritairement des batônnets. En conséquence, l'acuité visuelle et la perception des couleurs dans la vision périphérique diminuent avec son excentricité, mais celle-ci est par contre très sensible aux variations de luminance et de fréquence spatiale. \autocite{Werner2014}\\
Cette caractéristique de notre système visuel permet de fortement réduire la quantité d'informations que doivent traiter les réseaux nerveux, passant d'un flux arrivant à la rétine estimé à $10^{8}$ bits/s à une sortie par le nerf optique estimé à $10^{2}$ bits/s. \autocite{Kortum1996, Werner2014, Zhaoping2014}\\

Mais cette optimisation du flux d'informations présente au moins un inconvénient majeur. Une description précise d'un stimulus visuel ne peut être réalisée avec une certitude élevée que dans une partie très réduite du champs visuel (environ 2° chez l'Humain).\\
Ainsi lors de l'exploration visuelle de son environnement, un agent va devoir réaliser une suite de mouvements oculaires brefs afin de placer les cibles visuelles dans sa vision centrale et ainsi pouvoir en réaliser des descriptions précises.