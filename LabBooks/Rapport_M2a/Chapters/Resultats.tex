% Chapter Template

\chapter{Résultats} % Main chapter title
 
\label{Résultats} % Change X to a consecutive number; for referencing this chapter elsewhere, use \ref{ChapterX}

%----------------------------------------------------------------------------------------

\section{Apprentissage supervisé}

L'étape de benchmarking du paramètre d'apprentissage $\alpha$ (équation~\ref{eqn:Grad_desc}) permet de rendre compte de son importance sur l'efficacité de l'apprentissage et du modèle. \\
Dans le cadre du filtre \textit{Wavelets}, on peut ainsi observer qu'une valeur $\alpha\geq0.4$ entraîne un sur-apprentissage très important (le coût augmente fortement au cours de l'entraînement, figures~\ref{fig:benchmark_surApp1} et \ref{fig:benchmark_surApp2}), tandis que $\alpha=0.3$ semble représenter une valeur optimale pour l'apprentissage (figure~\ref{fig:benchmark_alpha}).\\
Dans le cadre du filtre \textit{LogPolar}, deux jeux de poids indépendants doivent être optimisés pendant l'apprentissage,  correspondant respectivement aux couches \textit{détecteur} et \textit{classifieur}. Chaque couche possède ainsi son propre paramètre $\alpha$ ($\alpha_{detect}$ et $\alpha_{classif}$) et l'on peut donc calculer leurs coûts indépendamments (figure~\ref{fig:logpolar_cost}).

%----------------------------------------------------------------------------------------

\section{Prédiction de la position}

Après entraînement, le modèle est capable de détecter la cible dans son environnement et de prédire précisemment sa position (figure~\ref{fig:saccades}). Il est ensuite capable d'utiliser cette prédiction pour réaliser une saccade les coordonnées prédites de la cible visuelle, ce qui modifie en conséquence sa perception de son environnement.\\
Mais une seule saccade n'est pas toujours suffisante pour atteindre la cible (:::figure:::) et le nombre de saccades nécessaires augmente avec la distance initiale de la cible à la fovéa (:::figure:::). Cette relation pourrait provenir de la diminution de l'acuité avec l'excentricité dans la champs visuel (provoquée par le champs rétinien), entraînant une diminution de la précision des prédictions (:::figure:::)

%----------------------------------------------------------------------------------------

\section{Prédiction de la catégorie}

%----------------------------------------------------------------------------------------
