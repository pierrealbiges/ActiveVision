% Chapter Template

\chapter{Matériel et méthodes} % Main chapter title
\label{Materiel_methode} % Change X to a consecutive number; for referencing this chapter elsewhere, use \ref{ChapterX}

%----------------------------------------------------------------------------------------

\section{Support physique et numérique} %Spécificités ordinateur/machine virtuelle, python/libraries versions
L'ensemble des modélisations ont été réalisés sur une ordinateur portable hébergeant une machine virtuelle. Leurs caractéristiques sont rassemblées dans le tableau suivant :\\

\resizebox{18.75cm}{!}{
\begin{tabular}{| p{4cm} || l | l | p{5cm} | l | l |}
\hline
& Identifiant & Système d'explotation & Processeur & Mémoire vive & Carte graphique\\ \hline
Machine physique & ASUS ROG G75VW & Windows 7 64-bit SP1 & Intel Core I7-3610QM 2,30GHz (8CPU) &  
8 GB (DDR3) & NVIDIA GeForce GTX670M\\ \hline
Machine virtuelle (ressources allouées) & VirtualBox v.5.2.6 & Ubuntu 16.04 & 4 CPU, 90\% des ressources & 
5298 Mo & Support GPU non-utilisé\\ \hline
\end{tabular}
}

Les modélisation ont été réalisées à l'aide du language de programmation \href{https://www.python.org/}{Python} (version 3.6.4) renforcé de la librairie \href{https://www.tensorflow.org/}{TensorFlow} (version 1.4) et de l'interface graphique \href{https://jupyter.org/}{Jupyter}.

%----------------------------------------------------------------------------------------

\section{Modèle POMDP} %Description du modèle perception-action, schéma explicatif

%----------------------------------------------------------------------------------------

\section{Régression linéaire multivariée} %Formules du modèle mathématiques soutenant l'apprentissage