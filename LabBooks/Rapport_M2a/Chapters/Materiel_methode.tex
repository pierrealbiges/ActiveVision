% Chapter Template

\chapter{Matériel et méthodes} % Main chapter title
\label{Materiel_methode} % Change X to a consecutive number; for referencing this chapter elsewhere, use \ref{ChapterX}

%----------------------------------------------------------------------------------------

\section{Support physique et numérique} %Spécificités ordinateur/machine virtuelle, python/libraries versions
L'ensemble des modélisations ont été réalisés sur une ordinateur portable hébergeant une machine virtuelle. Leurs caractéristiques sont rassemblées dans le tableau suivant :\\

\resizebox{19cm}{!}{
\begin{tabular}{| p{4cm} || l | l | p{5cm} | l | l |}
\hline
& Identifiant & Système d'explotation & Processeur & Mémoire vive & Carte graphique\\ \hline
Machine physique & ASUS ROG G75VW & Windows 7 64-bit SP1 & Intel Core I7-3610QM 2,30GHz (8CPU) &  
8 GB (DDR3) & NVIDIA GeForce GTX670M\\ \hline
Machine virtuelle (ressources allouées) & VirtualBox v.5.2.6 & Ubuntu 16.04 & 4 CPU, 90\% des ressources & 
5298 Mo & Support GPU non-utilisé\\ \hline
\end{tabular}
}

Les modélisation ont été réalisées à l'aide du language de programmation \href{https://www.python.org/}{Python} (version 3.6.4) renforcé de la librairie \href{https://www.tensorflow.org/}{TensorFlow} (version 1.4) et de l'interface graphique \href{https://jupyter.org/}{Jupyter}.\\
La base de données \href{http://yann.lecun.com/exdb/mnist/}{MNIST} a été utilisée pour l'apprentissage et l'évaluation du modèle. Elle contient 70.000 images de chiffres manuscrits (60.000 pour l'entraînement, 10.000 pour l'évaluation), centrés et dont la taille a été normalisée. Chaque image est accompagnée d'un label décrivant quel chiffre elle contient.

%----------------------------------------------------------------------------------------

\section{Modèle POMDP} %Description du modèle perception-action, schéma explicatif
Le problème de recherche d'information dans un contexte d'exploration de l'environnement visuel peut être formulé comme un \textbf{processus de décision Markovien partiellement observable} (POMDP) \autocite{Butko2010}. \\
Dans un POMDP, l'agent perçoit partiellement l'\textbf{état de l'environnement} \textit{S} à un temps \textit{t} (dans ce travail l'environnement visuel, perçut au travers d'un champs rétinien) et peut réaliser des \textbf{actions} \textit{A} (ici des saccades oculaires) qui peuvent avoir des conséquences l'environnement et sa perception \textit{O}. L'agent va ainsi construire un \textbf{état de croyance} \textit{B} (ici la catégorie prédite du stimulus) en fonction des observations et des actions réalisées jusqu'ici \autocite{Butko2010}.

\begin{figure}[th]
\centering
\includegraphics[scale=0.5]{Figures/POMDP}
\decoRule %puts an aesthetic horizontal line below the image
\caption[Figure]{Schéma des interations entre l'agent et son environnement au cours du temps dans un modèle POMDP}
\label{fig:POMDP}
\end{figure}

Un tel système doit satisfaire la \textbf{propriété de Markov}, qui décrit que la distribution de probabilité des futurs états ne dépends que de l'état précédent et pas de toute la séquence d'états les précédents.\\
Ainsi lors de l'évolution du système dans le temps, on considère que l'état suivant de l'environnement est uniquement influencé par son état actuel et l'action (éventuelle) réalisée par l'agent (équation~\ref{eqn:POMDP_sta}) \autocite{Butko2010}. 

\begin{equation}
p(s_{t+1}|s_{1:t},a_{1:t},o_{1:t}) = p(s_{t+1}|s_{t},a_{t})
\label{eqn:POMDP_sta}
\end{equation}

De même, les observations actuelles de l'agent ne dépendent que de l'état actuel de l'environnement et de l'action (éventuelle) qu'il réalise (équation~\ref{eqn:POMDP_obs}) \autocite{Butko2010}.

\begin{equation}
p(o_{t}|s_{1:t},a_{1:t}) = p(o_{t}|s_{t},a_{t})
\label{eqn:POMDP_obs}
\end{equation}

\begin{equation}
B_{t}^i = p(S_{t} = i|A_{1:t},O_{1:t})
\label{eqn:POMDP_bel}
\end{equation}


%----------------------------------------------------------------------------------------

\section{Champs rétinien} %Description du modèle logpolar

%----------------------------------------------------------------------------------------

\section{Apprentissage supervisé} %Formules du modèle mathématiques soutenant la Régression linéaire multivariée
                                                                    %Prétraitements de l'image (whitening, resize et déplacement cible

Afin d'obtenir un modèle à la fois performant et adaptable, nous l'avons soumis à un apprentissage supervisé sous la forme d'une \textbf{régression linéaire multivariée} optimisée par \textbf{descente de gradient}.\\
Pour cela, nous avons calculé une hypothèse (équation~\ref{eqn:Hypo}

\begin{equation}
h_{\theta}(x) = \theta^{T}x + b
\label{eqn:Hypo}
\end{equation}

\begin{equation}
J(\theta) = \frac{1}{2m} \sum_{i=1}^m (h_\theta(x^i)-y^i)^2
\label{eqn:Cost}
\end{equation}

\begin{equation}
\theta_j := \theta_j - \alpha \frac{1}{m} \sum_{i=1}^m (h_\theta(x^i) - y^i)x_{j}^i
\label{eqn:Grad_desc}
\end{equation}